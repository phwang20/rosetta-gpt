\documentclass[12pt]{book}
\usepackage[paperwidth=17in, paperheight=9in]{geometry}
\usepackage[osf,p]{libertinus}
\usepackage{microtype}
\usepackage[pdfusetitle,hidelinks]{hyperref}
\usepackage[series={},nocritical,noend,nofamiliar,noledgroup]{reledmac}
\usepackage{reledpar}
\setmainfont{Baskerville}

\usepackage{graphicx}
\usepackage{polyglossia}
\setmainlanguage{fioesjfsoeifj}
\setotherlanguage{chinese}
\usepackage{metalogo}

\linenumincrement*{1}
\firstlinenum*{1}
\setlength{\Lcolwidth}{0.48\textwidth}
\setlength{\Rcolwidth}{0.48\textwidth} 
\usepackage{xeCJK}
\setCJKmainfont{Songti TC}

\begin{document}

\title{燕丹子--fioesjfsoeifj Translation}
\date{}
\author{Translated by P.K. Hwang}

\maketitle\begin{pairs}
\begin{Rightside}
\begin{chinese}
\beginnumbering
\pstart
燕丹子質於秦,秦王遇之無禮,不得意,欲歸。秦王不聽,謬言曰:「令烏白頭,馬生角,乃可。」丹仰天歎,果烏白頭、馬生角。秦王不得已而遣之,為機發之橋,欲陷丹。丹過之,橋為不發。夜到關,關門未開。丹為鷄鳴,眾雞皆鳴,遂得逃歸。深怨於秦,求欲復之。奉養勇士,無所不至。
\pend
\endnumbering
\end{chinese}
\end{Rightside}
\begin{Leftside}
\begin{fioesjfsoeifj}
\beginnumbering
\pstart
Yan Dan, a man of Yan, went to Qin but was treated rudely by the king of Qin and was unhappy, so he wanted to return home. However, the Qin king refused to let him go and said falsely, ``I will release you if you can make a crow have a white head and a horse grow horns.'' Yan Dan looked up to the sky and sighed, then a crow with a white head and a horse with horns appeared miraculously.

The Qin king, with no other choice, sent Yan Dan away and ordered the construction of a bridge that would collapse under his weight, hoping to trap and kill Yan Dan. However, Yan Dan passed over the bridge safely, and it did not collapse. When he arrived at the city gate at night, it was closed, so Yan Dan imitated the crow's crowing, and all the roosters in the area crowed in response, causing the gatekeepers to open the gate. Yan Dan escaped and returned home, holding a deep grudge against Qin and seeking revenge.

He then gathered brave men and trained them to be formidable warriors, willing to do anything to achieve his revenge.
\pend
\endnumbering
\end{fioesjfsoeifj}
\end{Leftside}
\end{pairs}
\Columns
\begin{pairs}
\begin{Rightside}
\begin{chinese}
\beginnumbering
\pstart
為書與其傅鞠武曰:「不肖,生於僻陋之國,長於無毛之地,未嘗得覩君子雅訓、達人之道也。然鄙意欲有所陳,幸傅正覽之。丹聞丈夫所恥,恥受辱以生於世也;貞女所羞,羞見劫以虧其節也。故有刎喉不顧、據鼎不迴者,斯豈樂死而忘生哉,其心有所守也。今秦王反戾天常,虎狼其行,遇丹無禮,為諸侯最。丹每念之,痛入骨髓。計燕國之眾不能敵之,曠年相守,力固不足。欲收天下之勇士,集海內之英雄,破國空蔵,以奉養之,重幣甘辭以巿秦。貪我賂,而信我辭,一劍之任,可當百萬之師;須臾之間,可解丹萬世之恥。若其不然,令丹生無面目於天下,死懷恨於九泉,必令諸侯指以為笑,易水之北,未知誰有。此蓋亦子大夫之恥也。謹遣書,願熟之。」
\pend
\endnumbering
\end{chinese}
\end{Rightside}
\begin{Leftside}
\begin{fioesjfsoeifj}
\beginnumbering
\pstart
The following is a letter written by Yan Dan to his teacher Ju Wu:

``I am unworthy, born in a backward country and raised in a land without culture. I have never had the opportunity to learn from wise men or to understand the way of the virtuous. Nevertheless, I wish to express my thoughts and hope that you, my teacher, will kindly read them.

I have heard that a man's greatest shame is to be born into this world only to suffer humiliation, and that a woman's greatest disgrace is to be violated and lose her virtue. This is why there are those who would rather die than suffer such indignities, and they are not seeking death but rather are protecting what they hold dear in their hearts.

Now the King of Qin is going against the natural order of things and acting like a tiger or a wolf. He treated me rudely, which is the worst among all the lords. Whenever I think of it, it pains me to the core.

The strength of the Yan country is insufficient to resist Qin's power, and we have been at war for many years without success. I intend to gather the bravest warriors and heroes from all over the world and use our country's wealth to hire them to help us defeat Qin. They will be willing to fight even a million soldiers with just one sword, and within a short time, we can wipe away the shame of Yan Dan that has lasted for thousands of years.

If this plan fails, I will have no face left in this world and will die with a bitter heart in the underworld. The other lords will mock and laugh at me, and I will not know peace even in death. This would be a great shame for you, my teacher.

I send you this letter and hope you will carefully consider my proposal.''
\pend
\endnumbering
\end{fioesjfsoeifj}
\end{Leftside}
\end{pairs}
\Columns
\begin{pairs}
\begin{Rightside}
\begin{chinese}
\beginnumbering
\pstart
鞠武報書曰:「臣聞快於意者虧於行,甘於心者傷於性。今太子欲滅悁悁之恥,除久久之恨,此實臣所當麋軀碎首而不避也。私以為智者不冀僥倖以要功,明者不苟從志以順心。事必成然後舉,身必安而後行。故發無失舉之尤,動無蹉跌之愧也。太子貴匹夫之勇,信一劍之任,而欲望功,臣以為䟽。臣願合從於楚,并勢於趙,連衡於韓、魏,然後圖秦,秦可破也。且韓、魏與秦,外親內䟽。若無倡兵,楚乃來應,韓、魏必從,其勢可見。今臣計從,太子之恥除,愚鄙之累解矣。太子慮之。」
\pend
\endnumbering
\end{chinese}
\end{Rightside}
\begin{Leftside}
\begin{fioesjfsoeifj}
\beginnumbering
\pstart
In a letter to his tutor Ju Wu, the author wrote:

``Esteemed Sir, I am unworthy and was born in a backward country and raised in an uncivilized land, where I have never had the opportunity to learn from virtuous gentlemen or wise sages. However, I desire to express my thoughts and humbly ask for your guidance. I have heard that a man is ashamed to have been born if he suffers indignity in life, and a woman is ashamed to have lived if she loses her chastity. Thus, there are those who would rather die than lose their honor and integrity. Today, the King of Qin acts against the will of Heaven and is as cruel as a tiger or a wolf. He has treated me disrespectfully, making himself the most hated ruler among the vassals. Every time I think about this, it pains me to the core. The strength of the Yan state is not enough to withstand him, and we have been at a stalemate for years. Therefore, I want to gather the bravest warriors and heroes from all over the world and use their strength to defeat Qin. I will offer them ample rewards and persuade them to fight for us, even if it means spending a fortune to bribe them. If they take our money and believe our words, then a single warrior could defeat a million soldiers, and we could remove the eternal shame that weighs on me in an instant. If not, I will have no face to show in the world and will die with a grudge in my heart, while the vassals will mock me, and nobody will know my name north of the Yi River. This would be a great shame for me.

I am sending this letter to you, and I hope that you will consider it carefully.''

Ju Wu replied:

``I have heard that those who act impulsively are likely to make mistakes, and those who indulge their desires will harm their own nature. Your Highness wants to remove the shame that burdens you and eliminate the resentment that has accumulated over time. This is something that I, as your humble servant, would do my utmost to achieve, even if it means sacrificing my life. However, I believe that wise men do not rely on chance to achieve their goals, and enlightened ones do not blindly follow their own desires to satisfy their hearts. We should act only when success is assured, and we should move only when we are safe. This is the way to avoid mistakes and avoid humiliation.

Your Highness values the bravery of a common man and believes that one sword can accomplish great things. But I think this is an illusion. I am willing to unite with the Chu state, ally with the Zhao state, and form a coalition with the Han and Wei states to attack Qin. If we work together, we can defeat Qin. Furthermore, Han and Wei are both friendly with Qin on the surface but hostile underneath. If we launch an offensive, Chu will come to our aid, and Han and Wei will have to join us. This is the course of action that I propose, which will not only remove Your Highness's shame but also relieve me of my own petty concerns. Please consider my proposal.''
\pend
\endnumbering
\end{fioesjfsoeifj}
\end{Leftside}
\end{pairs}
\Columns
\begin{pairs}
\begin{Rightside}
\begin{chinese}
\beginnumbering
\pstart
太子得書,不說,召鞠武而問之,武曰:「臣以為太子行臣言,則易水之北,永無秦憂,四鄰諸侯必有求我者矣。」太子曰「此引日縵縵,心不能須也!」鞠武曰:「臣為太子計熟矣。夫有秦、疾不如徐,走不如坐。今合楚、趙,并韓、魏,雖引歲月,其事必成。臣以為良。」太子睡臥不聽。鞠武曰:「臣不能為太子計。臣所知田光,其人深中有謀。願令見太子。」太子曰:「敬諾!」
\pend
\endnumbering
\end{chinese}
\end{Rightside}
\begin{Leftside}
\begin{fioesjfsoeifj}
\beginnumbering
\pstart
After receiving the letter, the Crown Prince did not express his joy, but instead summoned Jiu Wu and asked for his opinion. Jiu Wu replied, ``I believe that if you follow my advice, then north of the Yi River, there will never be any more concerns from Qin, and the neighboring states will surely come to seek our alliance.'' The Crown Prince responded, ``This plan is too slow, and my heart cannot wait.'' Jiu Wu said, ``I have thought carefully about this plan. If we ally with Chu and Zhao, as well as Han and Wei, even if it takes years, the plan will surely succeed. I think it is a good plan.'' However, the Crown Prince did not listen to him and refused to sleep or rest. Jiu Wu then said, ``I cannot offer any more advice to the Crown Prince. I know a man named Tian Guang who has deep and clever ideas. I suggest that you meet with him.'' The Crown Prince agreed respectfully.\pend
\endnumbering
\end{fioesjfsoeifj}
\end{Leftside}
\end{pairs}
\Columns
\begin{pairs}
\begin{Rightside}
\begin{chinese}
\beginnumbering
\pstart
田光見太子,太子側階而迎,迎而再拜。坐定,太子丹曰:「傅不以蠻域而丹不肖,乃使先生來降弊邑,今燕國僻在北陲,比於蠻域,而先生不羞之。丹得侍左右,覩見玉顏,斯乃上世神靈,保祐燕國,令先生設降辱焉。」田光曰:「結髮立身,以至於今,徒慕太子之高行,美太子之令名耳。太子將何以教之?」太子膝行而前,涕淚橫流曰:「丹嘗質於秦,秦遇丹無禮,日夜焦心,思欲復之。論眾則秦多,計強則燕弱。欲曰合從,心復不能。常食不識位、寢不安席。縱令燕秦同日而亡,則為死灰復燃,白骨更生。願先生圖之。」田光曰:「此國事也,請得思之。」於是舍光上館。太子三時進食,存問不絕,如是三月。
\pend
\endnumbering
\end{chinese}
\end{Rightside}
\begin{Leftside}
\begin{fioesjfsoeifj}
\beginnumbering
\pstart
When Tian Guang met the Crown Prince, the Crown Prince greeted him from the side and bowed again after the greeting. After they sat down, the Crown Prince Dan said, ``My father did not appoint you to serve in a barbarian land, and I am not capable, yet you have come to lower yourself to serve in our humble state. Our state of Yan is located on the northern border and is comparable to a barbarian land, yet you do not feel ashamed. By serving in my court, you have the chance to see my handsome face, and this is because the divine beings of the past are protecting and blessing Yan. I humbly request that you offer your counsel to me.''

Tian Guang replied, ``I have maintained myself with the utmost integrity, and have always admired your high conduct and your virtuous name. But how can I teach you, Your Highness?'' The Crown Prince then kneeled forward and wept tears as he said, ``I was once a hostage in Qin, and Qin treated me disrespectfully. I have been consumed with rage day and night, and I wish to avenge the insult. If we compare the number of Qin's troops with ours, they have more, and if we compare our strength, Yan is weaker. If we want to unite with other states, my heart is not willing. I eat without knowing my position, and I cannot sleep peacefully. Even if Yan and Qin were to fall on the same day, it would be like a fire that reignites from the ashes and our bones would rise again. I hope that you can come up with a solution.''

Tian Guang replied, ``This is a matter of state, and I need to ponder on it.'' Then he retired to the guest house. The Crown Prince had three meals a day with Tian Guang, and he continued to ask questions without ceasing for three months.
\pend
\endnumbering
\end{fioesjfsoeifj}
\end{Leftside}
\end{pairs}
\Columns
\begin{pairs}
\begin{Rightside}
\begin{chinese}
\beginnumbering
\pstart
太子怪其無說,就光辟左右,問曰:「先生既垂哀恤,許惠嘉謀。側身傾聽,三月於斯,先生豈有意歟?」田光曰:「微太子言,固將竭之。臣聞騏驥之少,力輕千里,及其罷朽,不能取道。太子聞臣時已老矣。欲為太子良謀,則太子不能;欲奮筋力,則臣不能。然竊觀太子客無可用者,夏扶、血勇之人,怒而面赤;宋臆、脈勇之人,怒而面青;武陽,骨勇之人,怒而面白。光所知荊軻、神勇之人,怒而色不變,為人博聞強記,體烈骨壯,不拘小節,欲立大功。嘗家於衛,脫賢大夫之急十有餘人,其餘庸庸不可稱。太子欲圖事,非此人莫可。」太子下席再拜曰:「若因先生之靈,得交於荊君,則燕國社稷長為不滅。唯先生成之。」田光遂行。太子自送,執光手曰:「此國事,願勿洩之!」光笑曰:「諾。」
\pend
\endnumbering
\end{chinese}
\end{Rightside}
\begin{Leftside}
\begin{fioesjfsoeifj}
\beginnumbering
\pstart
The prince was puzzled by Tian Guang's lack of words, so he dismissed his attendants and asked him, ``Since you have shown compassion and offered wise counsel, have you been thinking about it during these three months?'' Tian Guang replied, ``Though Your Highness's words are humble, I will do my best to respond. I have heard that a fine steed when young may cover a thousand li with ease, but when it is old and weary, it cannot even take a single path. Your Highness has heard that I am already old. If Your Highness seeks good counsel, then I am unable to offer it; if Your Highness requires strength and power, then I am unable to provide it. However, I have observed that among Your Highness's attendants, there are none that can be trusted. The Summer and Blood Brave men grow red-faced with anger, while the Song and Pulse Brave men turn blue-faced with rage. Wu Yang is a Bone Brave man, but his face turns pale with fury. The only person I know who remains unchanged in the face of anger is Jing Ke, a brave and godly man who is widely learned, possesses great memory, is physically strong and resolute, and is not limited by small details when it comes to achieving great deeds. He has stayed in my house in Wei before, and he once saved more than ten virtuous gentlemen from a perilous situation, while the rest were mediocre and unworthy of mention. If Your Highness wishes to accomplish great things, then only he can help you.'' The prince got up from his seat and bowed again, saying, ``If I can use your spirit to establish a relationship with Lord Jing, then the Yan state will thrive forever. It all depends on you, sir.'' Tian Guang then departed, and the prince escorted him out, holding his hand and saying, ``This is a matter of state, please do not reveal it!'' Tian Guang smiled and replied, ``I won't.''
\pend
\endnumbering
\end{fioesjfsoeifj}
\end{Leftside}
\end{pairs}
\Columns
\begin{pairs}
\begin{Rightside}
\begin{chinese}
\beginnumbering
\pstart
遂見荊軻,曰:「光不自度,不肖,達足下於太子。夫燕太子、真天下之士也,傾心於足下,願足下勿疑焉。」荊軻曰:「有鄙志,常謂心合意等,沒身不顧,情有乖異,一毛不拔。今先生令交於太子,敬諾不違。」田光謂軻曰:「蓋聞士不為人所疑。太子送光之時,言此國事,願勿洩,此疑光也。是疑而生於世,光所羞也。」向軻吞舌而死。軻遂之燕。
\pend
\endnumbering
\end{chinese}
\end{Rightside}
\begin{Leftside}
\begin{fioesjfsoeifj}
\beginnumbering
\pstart
Then he met Jing Ke and said, ``I, Guang, am unremarkable and foolish, but I have brought you to the attention of the Crown Prince. The Crown Prince of Yan is a truly remarkable man and has wholeheartedly devoted himself to you. He wishes you not to doubt his sincerity.'' Jing Ke replied, ``I have humble aspirations and have always believed that if our hearts and intentions are in harmony, we will not hesitate to risk our lives for each other. If there is any discord, even the slightest, we will abandon each other. Now that you have entrusted me to the Crown Prince, I will keep my word with respect and will not fail you.''

Tian Guang said to Jing Ke, ``It is said that a true gentleman is not doubted by others. When the Crown Prince saw me off, he told me about the affairs of the state and asked me not to reveal them. This was a sign of his suspicion of me. If such a suspicion were to spread, I would be ashamed. '' At this point, Guang choked and died. Jing Ke then proceeded to Yan.
\pend
\endnumbering
\end{fioesjfsoeifj}
\end{Leftside}
\end{pairs}
\Columns
\begin{pairs}
\begin{Rightside}
\begin{chinese}
\beginnumbering
\pstart
荊軻之燕,太子自御,虛左,軻緩不讓。至,坐定,賓客滿坐,軻言曰:「田光褒揚太子仁愛之風,說太子不世之器,高行厲天,美聲盈耳。軻出衛都,望燕路,歷險不以為勤,望遠不以為遐。今太子禮之以舊故之恩,接之以新人之敬,所以不復讓者,士信於知己也。」太子曰:「田先生無恙乎?」軻曰:「光臨送軻之時,言太子戒以國事,恥丈夫而不見信,向軻吞舌而死矣。」太子驚愕失色,噓唏飲淚曰:「丹所以戒先生,豈疑先生哉?今先生自殺,亦令丹自棄於世矣!」茫然良久,不怡,民氐日置酒請軻,酒酣,太子起為壽。夏扶前曰:「聞事無鄉曲之譽,則未可與論行;馬無服輿之伎,則未可與稱良。今荊君遠至,將何以教太子?」欲微感之。軻曰:「士有超世之行者,不必合於鄉曲;馬有千里之相者,何必出於服輿。昔呂望當屠釣之時,天下之賤丈夫也;其遇文王,則為周師。騏驥之在鹽車,駑之下也;及遇伯樂,則有千里之功。如此在鄉曲而後發善,服輿而後別良哉!」夏扶問軻:「何以教太子?」軻曰:「將令燕繼召公之迹,追甘棠之化,高欲四三王,下欲六五霸。於君何如?」坐皆稱善。竟酒,無能屈。太子甚喜,自以得軻,永無秦憂。
\pend
\endnumbering
\end{chinese}
\end{Rightside}
\begin{Leftside}
\begin{fioesjfsoeifj}
\beginnumbering
\pstart
During Jing Ke's visit to Yan, the crown prince himself drove his carriage and made way for Jing Ke, who did not hurry and yielded slowly. Upon arrival, as they took their seats, the guests were all gathered, and Jing Ke said, ``Tian Guang praised the crown prince's benevolent and loving spirit, saying that the crown prince is a rare talent in the world, who is able to hold himself to high standards and command the respect of all. When I left Weidu, I traveled along the road to Yan, and encountered many difficulties, but did not think of them as hardships, nor did I consider the distance as far. The reason why I do not yield is because I have faith in my friends.'' The crown prince asked, ``Is Mr. Tian Guang well?'' Jing Ke replied, ``When Guang came to send me off, he told me about the state affairs that the crown prince had confided in him. He was ashamed to be a man who was not trusted, so he swallowed his tongue and died.'' The crown prince was shocked and upset, and exclaimed with tears, ``Did I really doubt Mr. Tian Guang? Now that Mr. Jing has committed suicide, I have also abandoned myself to the world!'' After a while, he became melancholy and unhappy. On the following day, the people of Yan held a feast in honor of Jing Ke. When the wine was flowing freely, the crown prince stood up and offered a toast. Xia Fuqian said, ``If a person does not have the reputation of their hometown, they cannot be considered as having accomplished much. If a horse does not have the skill to pull a carriage, it cannot be considered as a good horse. Now that Mr. Jing has come from afar, what can he teach the crown prince?'' He intended to subtly touch Jing Ke's heart. Jing Ke replied, ``There are men who have surpassed the world, who need not conform to their hometown's customs. There are horses that have traveled a thousand miles, who need not be trained for pulling carriages. In the past, Lü Wang was a lowly man who butchered fish, but when he met King Wen of Zhou, he became the commander of the Zhou army. Ki Ji was a weak horse that pulled salt carts, but when he met Bo Le, he became a great horse that could travel a thousand miles. It is only by starting from our hometowns that we can achieve greatness, and it is only by starting from pulling carriages that we can become excellent horses.'' Xia Fuqian asked Jing Ke, ``What can you teach the crown prince?'' Jing Ke replied, ``I will let Yan follow in the footsteps of Duke Jie, follow the example of Gan Tang, and aspire to be like the four or three kings, and strive to be like the six or five hegemons. What do you think, Your Highness?'' All the guests praised Jing Ke's words and toasted him. They drank until they were completely intoxicated, and no one could match Jing Ke's eloquence. The crown prince was overjoyed, thinking that with Jing Ke's assistance, he would never have to worry about the Qin state.
\pend
\endnumbering
\end{fioesjfsoeifj}
\end{Leftside}
\end{pairs}
\Columns
\begin{pairs}
\begin{Rightside}
\begin{chinese}
\beginnumbering
\pstart
後日與軻之東宮,臨池水而觀。軻拾瓦投龜,太子令人捧盤。荊軻,投盡復進。軻曰:「非為太子愛金也,但臂痛耳。」後復共乘千里馬。軻曰:「馬肝甚美。」太子即殺馬進肝。暨樊將軍得罪於秦,秦求之急,乃來歸太子。太子置酒華陽之臺。酒中,太子出美人能琴者。軻曰:「好手琴者!」太子即進之。軻曰:「但愛其手耳。」太子斷手,盛以玉盤奉之。太子常與軻同案而食,同床而寢。
\pend
\endnumbering
\end{chinese}
\end{Rightside}
\begin{Leftside}
\begin{fioesjfsoeifj}
\beginnumbering
\pstart
On a later day, they went to the Eastern Palace and stood by a pool of water. Jing Ke picked up a tile and threw it at a turtle, and the Crown Prince ordered someone to bring a plate. Jing Ke continued throwing until he ran out of tiles. He said, ``I am not doing this because the Crown Prince loves gold, but because my arm hurts.'' Later, they rode together on a thousand-li horse. Jing Ke said, ``Horse liver is very delicious.'' The Crown Prince then had the horse killed and the liver served.

When General Fan offended the state of Qin and was in great danger, he came to surrender to the Crown Prince. The Crown Prince held a banquet for him at the Tower of Huayang. During the banquet, the Crown Prince brought out a beautiful woman who was skilled at playing the zither. Jing Ke said, ``She plays the zither well!'' The Crown Prince then gave her to Jing Ke. Jing Ke said, ``I only love her hands.'' The Crown Prince then cut off her hands and presented them to Jing Ke on a jade plate. The Crown Prince and Jing Ke often ate at the same table and slept in the same bed.
\pend
\endnumbering
\end{fioesjfsoeifj}
\end{Leftside}
\end{pairs}
\Columns
\begin{pairs}
\begin{Rightside}
\begin{chinese}
\beginnumbering
\pstart
後日,軻從容曰:「軻侍太子,三年於斯矣,而太子遇軻甚厚,黃金投龜,千里馬肝,姬人好手,盛以玉盤。凡庸人當之,猶尚樂出尺寸之長,當犬馬之用。今軻常侍君子之側,聞烈士之節,死有輕於鴻毛,義有重於太山,但聞用之所在耳。太子幸教之。」太子歛袂,正色而言曰:「丹嘗遊秦,秦遇丹不道,丹恥與之俱生。今荊君不以丹不肖,降辱小國。今丹以社稷干長者,不知所謂。」軻曰:「今天下彊國莫彊於秦。今太子力不能威諸侯,諸侯未肯為太子用也。太子率燕國之眾而當之,猶使羊將狼,使狼追虎耳。」太子曰:「丹之憂計久,不知安出?」軻曰:「樊於期得罪於秦,秦求之急。又督亢之地,秦所貪也。今得樊於期首、督亢地圖,則事可成也。」太子曰:「若事可成,舉燕國而獻之,丹甘心焉。樊將軍以窮歸我,而丹賣之,心不善也。」軻默然不應,居五月,太子恐軻悔,見軻曰:「今秦已破燕國,兵臨燕,事已迫急。雖欲足下計,安施之?今欲先遣武陽,何如?」軻怒曰:「何太子所遣,往而不返者,豎子也!軻所以未行者,待吾客耳。」於是軻潛見樊於期,曰:「聞將軍得罪於秦,父母妻子皆見焚燒,求將軍邑萬戶、金千斤。實為將軍痛之。今有一言,除將軍之辱,解燕國之恥,將軍豈有意乎?」於期曰:「常念之,日夜飲淚,不知所出。荊君幸教,願聞命矣!」軻曰:「得將軍之首與燕督亢地圖,秦必喜。喜而見軻,軻將左手把其袖,右手揕其胸,數以負燕之罪,責以將軍之御,而燕國見陵雪,將軍積忿之怒除矣。」於期起,振腕執刀曰:「是於期日夜所欲,而今聞命矣!」於是自刎,頭墜背後,兩目不瞑。太子聞之,自駕馳往,伏於期屍而哭,悲不自勝。良久,無柰何,遂函盛於期首與督亢地圖,武陽為副。軻不擇日而發,太子與知謀者皆素衣冠送易水上。軻起為,歌曰:「風蕭蕭兮易水寒,壯士一去不復還。」高漸離擊筑,宋臆和之。為壯聲,皆淚流。二子行過,夏扶當車前刎頸以送。二子行過陽翟,軻買肉爭輕重,屠辱之,武陽欲擊,軻止之。
\pend
\endnumbering
\end{chinese}
\end{Rightside}
\begin{Leftside}
\begin{fioesjfsoeifj}
\beginnumbering
\pstart
At a later date, Ke Congrong said calmly, ``I have been serving the Crown Prince for three years now, and he has treated me very well. He has given me gifts of gold turtle figurines, the liver of a thousand-mile horse, and skilled women to serve me on a jade platter. Such things would be enough to please any ordinary person, but I, as a close attendant of a gentleman, have heard of the loyalty and devotion of heroes who value their lives as lightly as goose feathers and their principles as heavily as Mount Tai, but only in theory. It is up to the Crown Prince to put these values into practice.'' The Crown Prince furrowed his brows and said in a serious tone, ``Dan once visited Qin, but he was not treated with respect. Dan was ashamed to have been born in the same time and place as the ruler of Qin. Now Jing Lord is demoting Dan to a lesser state, and Dan does not know what to do to serve the country.'' Ke replied, ``Nowadays, no state is stronger than Qin. The Crown Prince's power is not yet enough to command the loyalty of other lords, who are not willing to serve him. Even if the Crown Prince leads the army of Yan, it would be like sending a lamb to lead wolves, or like a wolf chasing a tiger.'' The Crown Prince asked, ``Dan has been worrying about this for a long time. Do you have any solutions?'' Ke said, ``Fan Yuqi has offended Qin and is in great danger. He also controls the region of Kuang, which Qin desires. If we can obtain the head of Fan Yuqi and a map of the Kuang region, the problem can be solved.''

The Crown Prince said, ``If it can be done, we shall offer Yan state as tribute, and Dan will be pleased. General Fan returned empty-handed to me, and Dan's willingness to sell is not good.'' Ke Mo remained silent. After five months, the Crown Prince feared that Ke would regret his decision, and so he met with Ke and said, ``Now that Qin has defeated Yan, the army is at Yan's doorstep, and the situation is urgent. Even if you have a plan, how can you put it into action? Now, do you agree to send Wu Yang first?'' Ke angrily replied, ``Whoever the Crown Prince sends will not return. He is just a child! The reason I have not left yet is that I am waiting for my guest.'' Ke then secretly met with Fan Yuqi and said, ``I have heard that you offended Qin and your parents, spouse, and children were all burned alive. I am truly saddened by this. I have a suggestion that can remove your disgrace and lift the shame from Yan state. Would you be willing to hear it?'' Fan replied, ``I have been thinking about this day and night, shedding tears without knowing what to do. If Lord Jing can teach me, I would like to hear the plan!'' Ke said, ``If we can get your head and a map of the Yudukou area, Qin will be pleased. When they are happy and meet with me, I will grab their sleeve with my left hand, poke their chest with my right hand, and accuse them of their crimes against Yan while blaming you for leading them. Yan state will then be relieved of its suffering, and your anger will be appeased.'' Fan stood up, took out his sword, and said, ``This is what I have been waiting for day and night. I now hear your command!'' He then committed suicide by cutting his own throat, his head hanging down his back and his eyes wide open. When the Crown Prince heard the news, he drove himself to the scene, lay on Fan's body and cried uncontrollably, overcome with grief. After a while, they placed Fan's head and the map of Yudukou in a box, and Wu Yang accompanied Ke. The Crown Prince and his advisors, dressed in plain clothes, saw them off at Yishui. Ke sang, ``The wind is blowing coldly over the Yishui River. Once a brave man has gone, he will never return.'' Gao Jianli played the zither while Song Yehua sang in harmony, and they all shed tears. When the two left, Xia Fudang cut his neck in front of their carriage and sent them off. As they passed Yangdi, Ke bought meat and competed with others in weighing it. When they were being slaughtered, Wu Yang wanted to attack, but Ke stopped him.
\pend
\endnumbering
\end{fioesjfsoeifj}
\end{Leftside}
\end{pairs}
\Columns
\begin{pairs}
\begin{Rightside}
\begin{chinese}
\beginnumbering
\pstart
西入秦,至咸陽,國中庶子蒙白曰:「燕太子丹畏大王之威,今奉樊於期首與督亢地圖,願為北蕃臣妾。」秦王喜。百官陪位,陛戟數百,見燕使者。軻奉於期首,武陽奉地圖。鐘聲並發,群臣皆呼萬歲。武陽大恐,兩足不能相過,面如死灰色。秦王怪之。軻見請曰:「此北鄙小子,希覩天闕。願大王小假,令得畢辭。」秦王謂軻曰:「取圖來。」進,圖窮而匕首出。軻左把秦王袖,右揕其胸,數之曰:「足下負燕日久,貪暴海內,不知厭足。於期無罪而夷其族。軻將海內報讎。今燕王母病,與軻促期,從吾計即生,不從則死。」秦王曰:「今日之事,從子計耳!乞聽琴聲而死。」召姬人鼓琴,琴聲曰:「羅縠單衣,可掣而絕。八尺屏風,可超而越。鹿盧之劍,可負而拔。」軻不曉音。秦王從言,掣之絕,超屏風,負劍而走。軻拔匕首擿之,決秦王,刃入銅柱,火出。秦王還斷軻兩手。軻倨詈曰:「坐。吾輕易為豎子所期。燕國之不報,我事之不立哉!」
\pend
\endnumbering
\end{chinese}
\end{Rightside}
\begin{Leftside}
\begin{fioesjfsoeifj}
\beginnumbering
\pstart
Upon entering the Qin territory and reaching Xianyang, a high-ranking official of the state named Meng Bai reported to the king saying, ``The Crown Prince of Yan is afraid of your majesty's power and now presents the city of Fan and a map of the Dutong territory, hoping to become a vassal of the Northern Frontier.'' The Qin king was delighted.

All the officials gathered around, hundreds of flags lined the stairs, and the Yan envoy was presented. Ke brought forth the city of Fan, and Wuyang brought the map. The sound of the bell rang out, and all the officials shouted, ``Long live the king!''

Wuyang was very frightened and could not move his legs. His face turned ashen. The Qin king was surprised and asked Ke, ``Who is this boy? Why is he so scared?'' Ke stepped forward and said, ``He is just a young boy from the northern border who has never seen the grandeur of the imperial court. I ask that Your Majesty be kind and allow him to finish his message.'' The Qin king replied, ``Go and get the map.''

As he advanced, his plan was exposed, and he drew his dagger. Ke held the sleeve of the Qin king with his left hand and pointed the dagger at his chest with his right, rebuking him, ``You have long oppressed Yan and been greedy for power, not content with your dominion. You unjustly annihilated the family of Dutong. I will avenge them and seek revenge throughout the realm. Now the mother of the Yan king is sick, and she has made an agreement with me to either bring me success or let me die.'' The Qin king said, ``Today's matter is in your hands. I beg to hear the sound of the qin before I die.'' He summoned a female musician to play the qin, and the music went, ``With a single robe of shimmering silk, he can wring your neck; behind an eight-foot screen, he can step over it; with the Deer Carver sword, he can carry it away by the hilt.'' Ke did not understand the words. The Qin king followed the instructions and wrung his neck, stepped over the screen, and carried away the sword, but Ke drew his dagger and stabbed the king. The blade pierced through the bronze pillar and caused sparks to fly. The Qin king then had Ke's hands cut off. Ke arrogantly cursed, ``Sit down. I, Ke, easily made a fool out of someone who is as foolish as you. If Yan does not avenge me, my mission will not be fulfilled!''
\pend
\endnumbering
\end{fioesjfsoeifj}
\end{Leftside}
\end{pairs}
\Columns
\end{document}