\documentclass[12pt]{book}
\usepackage[paperwidth=17in, paperheight=9in]{geometry}
\usepackage[osf,p]{libertinus}
\usepackage{microtype}
\usepackage[pdfusetitle,hidelinks]{hyperref}
\usepackage[series={},nocritical,noend,nofamiliar,noledgroup]{reledmac}
\usepackage{reledpar}
\setmainfont{Baskerville}

\usepackage{graphicx}
\usepackage{polyglossia}
\setmainlanguage{Ancient Greek}
\setotherlanguage{English}
\usepackage{metalogo}

\linenumincrement*{1}
\firstlinenum*{1}
\setlength{\Lcolwidth}{0.48\textwidth}
\setlength{\Rcolwidth}{0.48\textwidth} 
\usepackage{xeCJK}
\setCJKmainfont{Songti TC}

\begin{document}

\title{7th Letter-Plato--English Translation}
\date{}
\author{Translated by GPT}

\maketitle\begin{pairs}
\begin{Rightside}
\begin{Ancient Greek}
\beginnumbering
\pstart
[7.323δ] Πλάτων τοῖς Δίωνος οἰκείοις τε καὶ ἑταίροις εὖ πράττειν.
\pend
\endnumbering
\end{Ancient Greek}
\end{Rightside}
\begin{Leftside}
\begin{English}
\beginnumbering
\pstart
"[7.323d] Plato to Dion's household and friends, do well."
\pend
\endnumbering
\end{English}
\end{Leftside}
\end{pairs}
\Columns
\begin{pairs}
\begin{Rightside}
\begin{Ancient Greek}
\beginnumbering
\pstart
ἐπεστείλατέ μοι νομίζειν δεῖν τὴν διάνοιαν ὑμῶν εἶναι τὴν αὐτὴν ἣν εἶχεν καὶ Δίων, καὶ δὴ καὶ κοινωνεῖν διεκελεύεσθέ [7.324α] μοι, καθ᾽ ὅσον οἷός τέ εἰμι ἔργῳ καὶ λόγῳ. ἐγὼ δέ, εἰ μὲν δόξαν καὶ ἐπιθυμίαν τὴν αὐτὴν ἔχετε ἐκείνῳ, σύμφημι κοινωνήσειν, εἰ δὲ μή, βουλεύσεσθαι πολλάκις. τίς δ᾽ ἦν ἡ ἐκείνου διάνοια καὶ ἐπιθυμία, σχεδὸν οὐκ εἰκάζων ἀλλ᾽ ὡς εἰδὼς σαφῶς εἴποιμ᾽ ἄν.	
\pend
\endnumbering
\end{Ancient Greek}
\end{Rightside}
\begin{Leftside}
\begin{English}
\beginnumbering
\pstart
"You ordered me to believe that your way of thinking was the same as that of Dion, and you urged me to share in it to the extent that I am able, both in action and in speech. If you have the same opinion and desire as he did, I agree to share in it, but if not, we will discuss it often. But what was his way of thinking and desire, I hesitate to guess, but as one who knows clearly, I would speak."
\pend
\endnumbering
\end{English}
\end{Leftside}
\end{pairs}
\Columns
\begin{pairs}
\begin{Rightside}
\begin{Ancient Greek}
\beginnumbering
\pstart
ὅτε γὰρ κατ᾽ ἀρχὰς εἰς Συρακούσας ἐγὼ ἀφικόμην, σχεδὸν ἔτη τετταράκοντα γεγονώς, Δίων εἶχε τὴν ἡλικίαν ἣν τὰ νῦν Ἱππαρῖνος γέγονεν, καὶ ἣν ἔσχεν [7.324β] τότε δόξαν, ταύτην καὶ διετέλεσεν ἔχων, Συρακοσίους οἴεσθαι δεῖν ἐλευθέρους εἶναι, κατὰ νόμους τοὺς ἀρίστους οἰκοῦντας· ὥστε οὐδὲν θαυμαστὸν εἴ τις θεῶν καὶ τοῦτον εἰς τὴν αὐτὴν δόξαν περὶ πολιτείας ἐκείνῳ γενέσθαι σύμφρονα ποιήσειεν. τίς δ᾽ ἦν ὁ τρόπος τῆς γενέσεως αὐτῆς, οὐκ ἀπάξιον ἀκοῦσαι νέῳ καὶ μὴ νέῳ, πειράσομαι δὲ ἐξ ἀρχῆς αὐτὴν ἐγὼ πρὸς ὑμᾶς διεξελθεῖν· ἔχει γὰρ καιρὸν τὰ νῦν.	
\pend
\endnumbering
\end{Ancient Greek}
\end{Rightside}
\begin{Leftside}
\begin{English}
\beginnumbering
\pstart
For when I arrived in Syracuse at the beginning, having already lived almost forty years, Dion was of the age that Hipparete now is and at that time he possessed the same reputation and held the same position. He believed that the Syracusans should be free and governed by their best laws, so it is not surprising that if anyone, even a god, had made him of the same mind regarding the government, he would have made a worthy contribution. But how did he come to this position? It is worth hearing, whether one is young or old, and I will try to explain it to you from the beginning, for there is time for it now.
\pend
\endnumbering
\end{English}
\end{Leftside}
\end{pairs}
\Columns
\begin{pairs}
\begin{Rightside}
\begin{Ancient Greek}
\beginnumbering
\pstart
Νέος ἐγώ ποτε ὢν πολλοῖς δὴ ταὐτὸν ἔπαθον· ᾠήθην, εἰ θᾶττον ἐμαυτοῦ γενοίμην κύριος, ἐπὶ τὰ κοινὰ τῆς πόλεως [7.324c] εὐθὺς ἰέναι. καί μοι τύχαι τινὲς τῶν τῆς πόλεως πραγμάτων τοιαίδε παρέπεσον. ὑπὸ πολλῶν γὰρ τῆς τότε πολιτείας λοιδορουμένης μεταβολὴ γίγνεται, καὶ τῆς μεταβολῆς εἷς καὶ πεντήκοντά τινες ἄνδρες προύστησαν ἄρχοντες, ἕνδεκα μὲν ἐν ἄστει, δέκα δ᾽ ἐν Πειραεῖ--περί τε ἀγορὰν ἑκάτεροι τούτων ὅσα τ᾽ ἐν τοῖς ἄστεσι διοικεῖν ἔδει--τριάκοντα δὲ πάντων [7.324d] ἄρχοντες κατέστησαν αὐτοκράτορες. τούτων δή τινες οἰκεῖοί τε ὄντες καὶ γνώριμοι ἐτύγχανον ἐμοί, καὶ δὴ καὶ παρεκάλουν εὐθὺς ὡς ἐπὶ προσήκοντα πράγματά με. καὶ ἐγὼ θαυμαστὸν οὐδὲν ἔπαθον ὑπὸ νεότητος· ᾠήθην γὰρ αὐτοὺς ἔκ τινος ἀδίκου βίου ἐπὶ δίκαιον τρόπον ἄγοντας διοικήσειν δὴ τὴν πόλιν, ὥστε αὐτοῖς σφόδρα προσεῖχον τὸν νοῦν, τί πράξοιεν.	
\pend
\endnumbering
\end{Ancient Greek}
\end{Rightside}
\begin{Leftside}
\begin{English}
\beginnumbering
\pstart
Once I was young and I suffered many of the same things. I thought that if I became a ruler, I would immediately go to the city's public affairs. Certain events of the city befell me. For when the then current political system was being criticized by many, a change occurred and one hundred and fifty men stood out as leaders of the change, eleven in the city and ten in Piraeus (both in charge of everything related to governing the cities), and thirty became autocrats of them all. Some of them were my acquaintances and friends, and they asked me immediately to help with matters that concerned me. I did not suffer anything extraordinary in my youth. I thought that they were leading a just life out of some kind of injustice and that they would rule the city in a just manner and I paid great attention to what they would do.
\pend
\endnumbering
\end{English}
\end{Leftside}
\end{pairs}
\Columns
\begin{pairs}
\begin{Rightside}
\begin{Ancient Greek}
\beginnumbering
\pstart
καὶ ὁρῶν δήπου τοὺς ἄνδρας ἐν χρόνῳ ὀλίγῳ χρυσὸν ἀποδείξαντας τὴν ἔμπροσθεν πολιτείαν--τά τε ἄλλα καὶ φίλον [7.324e] ἄνδρα ἐμοὶ πρεσβύτερον Σωκράτη, ὃν ἐγὼ σχεδὸν οὐκ ἂν αἰσχυνοίμην εἰπὼν δικαιότατον εἶναι τῶν τότε, ἐπί τινα τῶν πολιτῶν μεθ᾽ ἑτέρων ἔπεμπον, βίᾳ ἄξοντα ὡς ἀποθανούμενον, [7.325a] ἵνα δὴ μετέχοι τῶν πραγμάτων αὐτοῖς, εἴτε βούλοιτο εἴτε μή· ὁ δ᾽ οὐκ ἐπείθετο, πᾶν δὲ παρεκινδύνευσεν παθεῖν πρὶν ἀνοσίων αὐτοῖς ἔργων γενέσθαι κοινωνός--ἃ δὴ πάντα καθορῶν καὶ εἴ τιν᾽ ἄλλα τοιαῦτα οὐ σμικρά, ἐδυσχέρανά τε καὶ ἐμαυτὸν ἐπανήγαγον ἀπὸ τῶν τότε κακῶν.	
\pend
\endnumbering
\end{Ancient Greek}
\end{Rightside}
\begin{Leftside}
\begin{English}
\beginnumbering
\pstart
And I saw men in a short time demonstrate the excellence of the government that preceded, and among them was Socrates, an older man whom I would almost hesitate to name as the most just of those of that time. They sent him, with others, to force someone to die so that he could be a participant in their affairs, whether he wanted to or not. But he did not obey, and endangered himself by refusing to participate in their unjust actions. Seeing all this and other similar things has made me uneasy and caused me to reflect on the evils of that time.
\pend
\endnumbering
\end{English}
\end{Leftside}
\end{pairs}
\Columns
\begin{pairs}
\begin{Rightside}
\begin{Ancient Greek}
\beginnumbering
\pstart
Χρόνῳ δὲ οὐ πολλῷ μετέπεσε τὰ τῶν τριάκοντά τε καὶ πᾶσα ἡ τότε πολιτεία· πάλιν δὲ βραδύτερον μέν, εἷλκεν δέ με ὅμως ἡ [7.325b] περὶ τὸ πράττειν τὰ κοινὰ καὶ πολιτικὰ ἐπιθυμία. ἦν οὖν καὶ ἐν ἐκείνοις ἅτε τεταραγμένοις πολλὰ γιγνόμενα ἅ τις ἂν δυσχεράνειεν, καὶ οὐδέν τι θαυμαστὸν ἦν τιμωρίας ἐχθρῶν γίγνεσθαί τινών τισιν μείζους ἐν μεταβολαῖς· καίτοι πολλῇ γε ἐχρήσαντο οἱ τότε κατελθόντες ἐπιεικείᾳ. κατὰ δέ τινα τύχην αὖ τὸν ἑταῖρον ἡμῶν Σωκράτη τοῦτον δυναστεύοντές τινες εἰσάγουσιν εἰς δικαστήριον, ἀνοσιωτάτην αἰτίαν ἐπιβαλόντες [7.325c] καὶ πάντων ἥκιστα Σωκράτει προσήκουσαν· ὡς ἀσεβῆ γὰρ οἱ μὲν εἰσήγαγον, οἱ δὲ κατεψηφίσαντο καὶ ἀπέκτειναν τὸν τότε τῆς ἀνοσίου ἀγωγῆς οὐκ ἐθελήσαντα μετασχεῖν περὶ ἕνα τῶν τότε φευγόντων φίλων, ὅτε φεύγοντες ἐδυστύχουν αὐτοί. 	
\pend
\endnumbering
\end{Ancient Greek}
\end{Rightside}
\begin{Leftside}
\begin{English}
\beginnumbering
\pstart
But soon after, the affairs of both the Thirty and the polis as a whole declined. Nevertheless, my desire for participating in political and communal matters persisted, albeit more slowly. Much was happening during that turbulent time, and it was not surprising that some individuals suffered greater retribution during the changes. However, those who had recently come into power showed a certain degree of leniency. By chance, our companion Socrates was accused by some powerful individuals and brought before the court for a crime that was most unjust and least of all applicable to Socrates himself. They accused him of impiety, and although some voted in his favor, he was ultimately condemned and put to death for not being willing to join in an unjust act against a friend who was fleeing with him when they were both in danger.
\pend
\endnumbering
\end{English}
\end{Leftside}
\end{pairs}
\Columns
\begin{pairs}
\begin{Rightside}
\begin{Ancient Greek}
\beginnumbering
\pstart
Σκοποῦντι δή μοι ταῦτά τε καὶ τοὺς ἀνθρώπους τοὺς πράττοντας τὰ πολιτικά, καὶ τοὺς νόμους γε καὶ ἔθη, ὅσῳ μᾶλλον διεσκόπουν ἡλικίας τε εἰς τὸ πρόσθε προύβαινον, τοσούτῳ χαλεπώτερον ἐφαίνετο ὀρθῶς εἶναί μοι τὰ πολιτικὰ [7.325d] διοικεῖν· οὔτε γὰρ ἄνευ φίλων ἀνδρῶν καὶ ἑταίρων πιστῶν οἷόν τ᾽ εἶναι πράττειν--οὓς οὔθ᾽ ὑπάρχοντας ἦν εὑρεῖν εὐπετές, οὐ γὰρ ἔτι ἐν τοῖς τῶν πατέρων ἤθεσιν καὶ ἐπιτηδεύμασιν ἡ πόλις ἡμῶν διῳκεῖτο, καινούς τε ἄλλους ἀδύνατον ἦν κτᾶσθαι μετά τινος ῥᾳστώνης--τά τε τῶν νόμων γράμματα καὶ ἔθη διεφθείρετο καὶ ἐπεδίδου θαυμαστὸν ὅσον, ὥστε με, [7.325e] τὸ πρῶτον πολλῆς μεστὸν ὄντα ὁρμῆς ἐπὶ τὸ πράττειν τὰ κοινά, βλέποντα εἰς ταῦτα καὶ φερόμενα ὁρῶντα πάντῃ πάντως, τελευτῶντα ἰλιγγιᾶν, καὶ τοῦ μὲν σκοπεῖν μὴ ἀποστῆναι μή ποτε ἄμεινον ἂν γίγνοιτο περί τε αὐτὰ ταῦτα καὶ [7.326a] δὴ καὶ περὶ τὴν πᾶσαν πολιτείαν, τοῦ δὲ πράττειν αὖ περιμένειν ἀεὶ καιρούς, τελευτῶντα δὲ νοῆσαι περὶ πασῶν τῶν νῦν πόλεων ὅτι κακῶς σύμπασαι πολιτεύονται--τὰ γὰρ τῶν νόμων αὐταῖς σχεδὸν ἀνιάτως ἔχοντά ἐστιν ἄνευ παρασκευῆς θαυμαστῆς τινος μετὰ τύχης--λέγειν τε ἠναγκάσθην, ἐπαινῶν τὴν ὀρθὴν φιλοσοφίαν, ὡς ἐκ ταύτης ἔστιν τά τε πολιτικὰ δίκαια καὶ τὰ τῶν ἰδιωτῶν πάντα κατιδεῖν· κακῶν οὖν οὐ [7.326b] λήξειν τὰ ἀνθρώπινα γένη, πρὶν ἂν ἢ τὸ τῶν φιλοσοφούντων ὀρθῶς γε καὶ ἀληθῶς γένος εἰς ἀρχὰς ἔλθῃ τὰς πολιτικὰς ἢ τὸ τῶν δυναστευόντων ἐν ταῖς πόλεσιν ἔκ τινος μοίρας θείας ὄντως φιλοσοφήσῃ.	
\pend
\endnumbering
\end{Ancient Greek}
\end{Rightside}
\begin{Leftside}
\begin{English}
\beginnumbering
\pstart
As I consider these things, and those who engage in politics, and laws and customs, the more I advance in age, the more difficult it seems to me to govern properly. For it is not possible to engage in these things without faithful friends and companions, whom it was not easy to find, for our city was no longer guided by the customs and practices of our ancestors, and it was impossible to gain new ones with ease. And the laws and customs were corrupted and distorted to an astonishing degree, so that when I saw the common good being pursued, and observed these things happening everywhere, I was filled with dread, and did not wish to turn away from considering them so that something better might come about regarding these things and indeed the entire state. But to act, I always waited for opportune times, and when they occurred, I made use of them to the best of my ability, and when they did not occur, I observed closely the fact that all cities nowadays are governed unjustly, because their laws nearly always lack some necessary provision, and there is no wonder-working chance to provide them. And so I am forced to speak in praise of true philosophy, since it is only through this that one can perceive what is just in politics and in all private affairs. It seems that human troubles will not come to an end until either those who practice true and genuine philosophy come to power in the state, or those who hold power in the state become true and genuine philosophers through some divine intervention.
\pend
\endnumbering
\end{English}
\end{Leftside}
\end{pairs}
\Columns
\begin{pairs}
\begin{Rightside}
\begin{Ancient Greek}
\beginnumbering
\pstart
Ταύτην δὴ τὴν διάνοιαν ἔχων εἰς Ἰταλίαν τε καὶ Σικελίαν ἦλθον, ὅτε πρῶτον ἀφικόμην. ἐλθόντα δέ με ὁ ταύτῃ λεγόμενος αὖ βίος εὐδαίμων, Ἰταλιωτικῶν τε καὶ Συρακουσίων τραπεζῶν πλήρης, οὐδαμῇ οὐδαμῶς ἤρεσεν, δίς τε τῆς ἡμέρας ἐμπιμπλάμενον ζῆν καὶ μηδέποτε κοιμώμενον μόνον νύκτωρ, [7.326c] καὶ ὅσα τούτῳ ἐπιτηδεύματα συνέπεται τῷ βίῳ· ἐκ γὰρ τούτων τῶν ἐθῶν οὔτ᾽ ἂν φρόνιμος οὐδείς ποτε γενέσθαι τῶν ὑπὸ τὸν οὐρανὸν ἀνθρώπων ἐκ νέου ἐπιτηδεύων δύναιτο--οὐχ οὕτως θαυμαστῇ φύσει κραθήσεται--σώφρων δὲ οὐδ᾽ ἂν μελλήσαι ποτὲ γενέσθαι, καὶ δὴ καὶ περὶ τῆς ἄλλης ἀρετῆς ὁ αὐτὸς λόγος ἂν εἴη, πόλις τε οὐδεμία ἂν ἠρεμήσαι κατὰ νόμους οὐδ᾽ οὑστινασοῦν ἀνδρῶν οἰομένων ἀναλίσκειν μὲν δεῖν [7.326d] πάντα εἰς ὑπερβολάς, ἀργῶν δὲ εἰς ἅπαντα ἡγουμένων αὖ δεῖν γίγνεσθαι πλὴν ἐς εὐωχίας καὶ πότους καὶ ἀφροδισίων σπουδὰς διαπονουμένας· ἀναγκαῖον δὲ εἶναι ταύτας τὰς πόλεις τυραννίδας τε καὶ ὀλιγαρχίας καὶ δημοκρατίας μεταβαλλούσας μηδέποτε λήγειν, δικαίου δὲ καὶ ἰσονόμου πολιτείας τοὺς ἐν αὐταῖς δυναστεύοντας μηδ᾽ ὄνομα ἀκούοντας ἀνέχεσθαι. 	
\pend
\endnumbering
\end{Ancient Greek}
\end{Rightside}
\begin{Leftside}
\begin{English}
\beginnumbering
\pstart
Having this mindset, I came to Italy and Sicily when I first arrived. But when I encountered what is called the "good life" here, full of Italian and Syracusan banquets, I did not like it at all. Living a life filled twice a day with food, never sleeping alone at night, and all the other habits that come with this lifestyle, no one could become wise pursuing them from a young age. One cannot be praised for such an abnormal habit. A wise person would never try to acquire such habits. And this same argument would also apply to other virtues. No city can remain peaceful according to laws, nor can any man believe that everything ought to be spent in excess, and at the same time, engage in idleness except for indulging in pleasures, drinks, and activities that cause distress. It is necessary for these cities to change their forms of government from tyranny, oligarchy, and democracy to a just and equal one, and not tolerate those who hold power in them, whether they hear or not the name of democracy.
\pend
\endnumbering
\end{English}
\end{Leftside}
\end{pairs}
\Columns
\begin{pairs}
\begin{Rightside}
\begin{Ancient Greek}
\beginnumbering
\pstart
Ταῦτα δὴ πρὸς τοῖς πρόσθε διανοούμενος, εἰς Συρακούσας [7.326e] διεπορεύθην, ἴσως μὲν κατὰ τύχην, ἔοικεν μὴν τότε μηχανωμένῳ τινὶ τῶν κρειττόνων ἀρχὴν βαλέσθαι τῶν νῦν γεγονότων πραγμάτων περὶ Δίωνα καὶ τῶν περὶ Συρακούσας· δέος δὲ μὴ καὶ πλειόνων ἔτι, ἐὰν μὴ νῦν ὑμεῖς ἐμοὶ πείθησθε τὸ δεύτερον συμβουλεύοντι. πῶς οὖν δὴ λέγω πάντων [7.327a] ἀρχὴν γεγονέναι τὴν τότε εἰς Σικελίαν ἐμὴν ἄφιξιν; ἐγὼ συγγενόμενος Δίωνι τότε νέῳ κινδυνεύω, τὰ δοκοῦντα ἐμοὶ βέλτιστα ἀνθρώποις εἶναι μηνύων διὰ λόγων καὶ πράττειν αὐτὰ συμβουλεύων, ἀγνοεῖν ὅτι τυραννίδος τινὰ τρόπον κατάλυσιν ἐσομένην μηχανώμενος ἐλάνθανον ἐμαυτόν. δίων μὲν γὰρ δή, μάλ᾽ εὐμαθὴς ὢν πρός τε τἆλλα καὶ πρὸς τοὺς τότε ὑπ᾽ ἐμοῦ λόγους γενομένους, οὕτως ὀξέως ὑπήκουσεν [7.327b] καὶ σφόδρα, ὡς οὐδεὶς πώποτε ὧν ἐγὼ προσέτυχον νέων, καὶ τὸν ἐπίλοιπον βίον ζῆν ἠθέλησεν διαφερόντως τῶν πολλῶν Ἰταλιωτῶν τε καὶ Σικελιωτῶν, ἀρετὴν περὶ πλείονος ἡδονῆς τῆς τε ἄλλης τρυφῆς ἠγαπηκώς· ὅθεν ἐπαχθέστερον τοῖς περὶ τὰ τυραννικὰ νόμιμα ζῶσιν ἐβίω μέχρι τοῦ θανάτου τοῦ περὶ Διονύσιον γενομένου.	
\pend
\endnumbering
\end{Ancient Greek}
\end{Rightside}
\begin{Leftside}
\begin{English}
\beginnumbering
\pstart
So thinking about these things, I journeyed to Syracuse, perhaps by chance, but it seems that at that time some of the better men had started the plot concerning Dion and the affairs of Syracuse; and indeed, even now, there is yet more fear, unless you now listen to my second counsel. How then, I ask, did my arrival in Sicily become the beginning of all this? For associating with Dion at that time, I exposed myself to new dangers, speaking and advising what seemed to me best for men to do, but not realizing that I was inadvertently planning the downfall of a tyranny. And as for Dion, because he was very intelligent in all matters, both in general and in particular those that I advised him on at that time, he obeyed me very sharply and eagerly, more than any young man I ever knew, and he wished to live the rest of his life quite differently than most of the Italians and Sicilians, having chosen virtue over most pleasures and other forms of luxury. Therefore, those who live under tyrannical laws had a more difficult life until the death of Dionysius.
\pend
\endnumbering
\end{English}
\end{Leftside}
\end{pairs}
\Columns
\begin{pairs}
\begin{Rightside}
\begin{Ancient Greek}
\beginnumbering
\pstart
Μετὰ δὲ τοῦτο διενοήθη μὴ μόνον ἐν αὑτῷ ποτ᾽ ἂν γενέσθαι ταύτην τὴν διάνοιαν, ἣν [7.327c] αὐτὸς ὑπὸ τῶν ὀρθῶν λόγων ἔσχεν, ἐγγιγνομένην δὲ αὐτὴν καὶ ἐν ἄλλοις ὁρῶν κατενόει, πολλοῖς μὲν οὔ, γιγνομένην δ᾽ οὖν ἔν τισιν, ὧν καὶ Διονύσιον ἡγήσατο ἕνα γενέσθαι τάχ᾽ ἂν συλλαμβανόντων θεῶν, γενομένου δ᾽ αὖ τοῦ τοιούτου τόν τε αὐτοῦ βίον καὶ τὸν τῶν ἄλλων Συρακουσίων ἀμήχανον ἂν μακαριότητι συμβῆναι γενόμενον. πρὸς δὴ τούτοις ᾠήθη δεῖν ἐκ παντὸς τρόπου εἰς Συρακούσας ὅτι τάχιστα ἐλθεῖν ἐμὲ [7.327d] κοινωνὸν τούτων, μεμνημένος τήν τε αὑτοῦ καὶ ἐμὴν συνουσίαν ὡς εὐπετῶς ἐξηργάσατο εἰς ἐπιθυμίαν ἐλθεῖν αὐτὸν τοῦ καλλίστου τε καὶ ἀρίστου βίου· ὃ δὴ καὶ νῦν εἰ διαπράξαιτο ἐν Διονυσίῳ ὡς ἐπεχείρησε, μεγάλας ἐλπίδας εἶχεν ἄνευ σφαγῶν καὶ θανάτων καὶ τῶν νῦν γεγονότων κακῶν βίον ἂν εὐδαίμονα καὶ ἀληθινὸν ἐν πάσῃ τῇ χώρᾳ κατασκευάσαι.	
\pend
\endnumbering
\end{Ancient Greek}
\end{Rightside}
\begin{Leftside}
\begin{English}
\beginnumbering
\pstart
After this, he considered that this way of thinking he had received from the correct reasoning was not only his own, but could also be found in others. However, it was not common, and only appeared in some, such as Dionysius, who thought it possible that one could become a god if caught by the gods. And if such a person should exist, it would be impossible for him not to achieve happiness, not only for himself but also for all the other Syracusans. Therefore, he thought he must go to Syracuse as soon as possible and become a part of this great and noble life. He remembered his past encounters with Dionysius and how he had expressed a desire to join this way of life. He had great hopes, without sacrifice, death, or the current evils befalling him, that he could create a happy and true life throughout the entire region.
\pend
\endnumbering
\end{English}
\end{Leftside}
\end{pairs}
\Columns
\begin{pairs}
\begin{Rightside}
\begin{Ancient Greek}
\beginnumbering
\pstart
Ταῦτα Δίων ὀρθῶς διανοηθεὶς ἔπεισε μεταπέμπεσθαι Διονύσιον ἐμέ, καὶ αὐτὸς ἐδεῖτο πέμπων ἥκειν ὅτι τάχιστα ἐκ [7.327e] παντὸς τρόπου, πρίν τινας ἄλλους ἐντυχόντας Διονυσίῳ ἐπ᾽ ἄλλον βίον αὐτὸν τοῦ βελτίστου παρατρέψαι. λέγων δὲ τάδε ἐδεῖτο, εἰ καὶ μακρότερα εἰπεῖν. τίνας γὰρ καιρούς, ἔφη, μείζους περιμενοῦμεν τῶν νῦν παραγεγονότων θείᾳ τινὶ τύχῃ; καταλέγων δὲ τήν τε ἀρχὴν τῆς Ἰταλίας καὶ Σικελίας [7.328a] καὶ τὴν αὑτοῦ δύναμιν ἐν αὐτῇ, καὶ τὴν νεότητα καὶ τὴν ἐπιθυμίαν τὴν Διονυσίου, φιλοσοφίας τε καὶ παιδείας ὡς ἔχοι σφόδρα λέγων, τούς τε αὑτοῦ ἀδελφιδοῦς καὶ τοὺς οἰκείους ὡς εὐπαράκλητοι εἶεν πρὸς τὸν ὑπ᾽ ἐμοῦ λεγόμενον ἀεὶ λόγον καὶ βίον, ἱκανώτατοί τε Διονύσιον συμπαρακαλεῖν, ὥστε εἴπερ ποτὲ καὶ νῦν ἐλπὶς πᾶσα ἀποτελεσθήσεται τοῦ τοὺς αὐτοὺς φιλοσόφους τε καὶ πόλεων ἄρχοντας μεγάλων [7.328b] συμβῆναι γενομένους. τὰ μὲν δὴ παρακελεύματα ἦν ταῦτά τε καὶ τοιαῦτα ἕτερα πάμπολλα,	
\pend
\endnumbering
\end{Ancient Greek}
\end{Rightside}
\begin{Leftside}
\begin{English}
\beginnumbering
\pstart
Having thought about these things correctly, Dion convinced me to send for Dionysios and he himself was requesting his arrival as soon as possible, so that before anyone else could persuade Dionysios to a different way of life, he could be directed towards a better one. He said he wanted to discuss many things, even if it took longer. "After all," he said, "what greater opportunities are we waiting for by some divine luck than those that have already come?". He talked about the beginning of Italy and Sicily, his own power in it, and the youth and desire for philosophy and education of Dionysios. He said that his siblings and relatives would make good companions for the conversation and way of life that I always speak of, and that they are most capable of encouraging Dionysios, so that if all goes well, we may hope that these same philosophers and rulers of great cities will be born. These were some of the instructions he had, among many others.
\pend
\endnumbering
\end{English}
\end{Leftside}
\end{pairs}
\Columns
\begin{pairs}
\begin{Rightside}
\begin{Ancient Greek}
\beginnumbering
\pstart
Τὴν δ᾽ ἐμὴν δόξαν τὸ μὲν περὶ τῶν νέων, ὅπῃ ποτὲ γενήσοιτο, εἶχεν φόβος--αἱ γὰρ ἐπιθυμίαι τῶν τοιούτων ταχεῖαι καὶ πολλάκις ἑαυταῖς ἐναντίαι φερόμεναι--τὸ δὲ Δίωνος ἦθος ἠπιστάμην τῆς ψυχῆς πέρι φύσει τε ἐμβριθὲς ὂν ἡλικίας τε ἤδη μετρίως ἔχον.
\pend
\endnumbering
\end{Ancient Greek}
\end{Rightside}
\begin{Leftside}
\begin{English}
\beginnumbering
\pstart
My opinion was that there was fear regarding the young people, regarding where they would end up, because their desires are swift and often in conflict with themselves; however, I perceived that Dion's character was of a gentle nature and his soul was entrenched in reason, having already reached a moderate age.
\pend
\endnumbering
\end{English}
\end{Leftside}
\end{pairs}
\Columns
\end{document}